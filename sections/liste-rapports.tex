\begin{consignes}
  une liste de rapporteurs potentiels et leur CV:
  \begin{itemize}
\item comprenant au minimum quatre noms choisis parmi les personnalités
  reconnues dans le domaine de recherche, habilitées à diriger des recherches, non
  membres du Cnam et non membres de l’ED concernée, et devant ne pas avoir
  publié récemment avec le candidat.
\item trois personnalités doivent être des professeurs des universités ou assimilés ou
  des personnalités étrangères exerçant des fonctions équivalentes dans leur pays.
\item Il est recommandé de proposer un ou deux rapporteurs exerçant des fonctions
  d’enseignement ou de recherche équivalentes à celles de professeur des
  universités dans un établissement étranger d’enseignement supérieur et de
  recherche.
\item Pour les candidats extérieurs au Cnam, les rapporteurs ne doivent pas appartenir
  à leur établissement de rattachement ni au Cnam.
\item un courrier du garant justifiant le choix des rapporteurs.
\item La liste des rapporteurs potentiels tend dans la mesure du possible vers la parité.
\item le garant doit également informer les rapporteurs pressentis de leur éventuelle
  participation à la procédure HDR.
  \end{itemize}
\end{consignes}

Voici la liste
