%\let\stdsection\section
%\renewcommand\section{\clearpage\stdsection}

%%%% debut macro %%%%
\def\keepspace{\ifnum\catcode`\ =10
  \let\next\keepspacebis \else \let\next\relax \fi
  \next}
\def \keepspacebis{\obeyspaces
  \afterassignment\keepspaceaux\let\next= }
{\obeyspaces%
\gdef\keepspaceaux{%
\ifx \next\space\let\next\ignorespaces\fi%
\catcode`\  =10\relax\next}}
%%%%
%%%% fin macro %%%%


\newcommand{\tbd}{{\bf TODO}\xspace}
\newcommand{\ie}{\emph{i.e.}\xspace}
\newcommand{\cad}{c'est-à-dire\xspace}
\newcommand{\via}{\emph{via}\xspace}
\newcommand{\apriori}{\emph{a priori}\xspace}
\newcommand{\aposteriori}{\emph{a posteriori}\xspace}
\newcommand{\cf}{\emph{cf.}\xspace}
\newcommand{\na}{\mbox{n/a}\xspace}
\newcommand{\ptop}{P2P\xspace}
\newcommand{\eg}{\emph{e.g.}\xspace}
\newcommand{\superscript}[1]{\ensuremath{^{\textrm{#1}}}}
\newcommand{\subscript}[1]{\ensuremath{_{\textrm{#1}}}}
\newcommand{\vs}{{\em vs}\xspace}
\newcommand{\prev}{{\em précédente}\xspace}
\newcommand{\suiv}{{\em suivante}\xspace}
\DeclareUrlCommand\printtopic{\urlstyle{tt}}
\newcommand{\topic}[1]{\protect\printtopic{#1}\index{#1}\xspace}
\newcommand{\urlfoot}[1]{\footnote{Voir~: \url{#1}.}\xspace}

\newcommand{\ecite}{{\em \tt \small (citation manquante)}\xspace}
\newcommand{\tocite}[1]{{\em \tt \small [? #1]}\xspace}

\newcommand{\fixthis}{{\em \bf (FIXTHIS)}\xspace}
\newcommand{\modif}{{\em \bf (FIXTHIS)}\xspace}
\newcommand{\fonction}[5]{\begin{array}{l|rcl}
#1: & #2 & \longrightarrow & #3 \\
    & #4 & \longmapsto & #5 \end{array}
}
\newcommand{\fonc}[3]{\begin{array}{llcl}
#1: & #2 & \longmapsto & #3 \end{array}
}

\usepackage{pgffor, ifthen}
\newcommand{\notes}[3][\empty]{%
  \noindent Notes\vspace{10pt}\\
  \foreach \n in {1,...,#2}{%
    \ifthenelse{\equal{#1}{\empty}}
      {\rule{#3}{0.5pt}\\}
      {\rule{#3}{0.5pt}\vspace{#1}\\}
    }
}

\newcommand{\includegraphicsmaybe}[2][]{\IfFileExists{#2}{\includegraphics{#2}}{\fbox{#2 is missing}\\}}
\newcommand\Mark[1]{\textsuperscript#1}

\newcommand{\bagwords}{\og bag-of-words \fg{}}
\newcommand{\weighting}{weighting function}
\newcommand{\weightings}{weighting functions}
\DeclareMathOperator*{\argmax}{arg\,max}
\DeclareMathOperator*{\aggr}{aggr}
\newcommand{\R}{\mathbb{R}} 
\newcommand{\card}{\mathrm{Card}}

%%%%%%%%%%% spécifique %%%%%%%%%
\newcommand{\supports}{\textsuperscript{\ding{72}}}
\newcommand{\logiciels}{\textsuperscript{\ding{73}}}

% affichage des heures ou pas 
\newcommand{\heures}[1]{}

% récap des publis (partagé)
\newcommand{\recapPublis}{%
\begin{center}\small
	\begin{tabular}{|l|l|c|}\hline
		\textbf{Audience} & \textbf{Catégorie} & \textbf{Nombre}\\\hline
		\multirow{4}{*}{Internationale} & Revue avec comité de rédaction      & {\color{White}0}1\\
		                                & Chapitres d'ouvrages avec comité de rédaction               & {\color{White}0}2\\
		                                & Conférences avec comité de sélection & {\color{White}0}9\\
		                                & \emph{Workshops} avec comité de sélection  & {\color{White}0}3\\\hline
		\multirow{3}{*}{Nationale}      & Revues avec comité de rédaction     & {\color{White}0}4\\
		                                & Conférences avec comité de sélection & 13\\
		                                & Atelier avec comité de sélection & {\color{White}0}1\\\hline
		\rowcolor[gray]{.9}\multicolumn{2}{|l|}{\textbf{Total des publications significatives}} & \textbf{33}\\\hline
		\multicolumn{2}{|l|}{Autres travaux}                        & {\color{White}0}3\\\hline
		\rowcolor[gray]{.9}\multicolumn{2}{|l|}{\textbf{Total des publications}} & \textbf{36}\\\hline
	\end{tabular}
\end{center}}

% récap des enseignements (partagé) \columncolor[gray]{.9}
\newcommand{\recapEnseignementsLight}{%
\begin{center}\small %>{\centering\bf\arraybackslash\columncolor[gray]{.9}}
\begin{tabular}{|l|l|l|*{4}{>{\centering\arraybackslash}p{1.3cm}|}|>{\centering\bf\arraybackslash\columncolor[gray]{.9}}p{1.1cm}|}\hline
\multirow{2}{*}{\textbf{Thème des enseignements}} & \multirow{2}{*}{\textbf{Niveau}} & \multirow{2}{*}{\textbf{Type}} & 
\multicolumn{3}{c|}{\textbf{Moniteur}} & \textbf{\nicefrac{1}{2} ATER} & 
\\\cline{4-7}%\hhline{~~----~}
&  & & \textbf{05--06} & \textbf{06--07}     & \textbf{07--08} & \textbf{08--09} & \multirow{-2}{*}{\textbf{Total}}\\\hline
Programmation orientée objet\supports{}         & L3      & TD, TP & 11     & 11   & 46   & 46    & 114\\
Bases de données\supports{} \& XML              & L2 $\to$ M1  & TP     & 25,5   & 35   & 8    & 39,5  & 108\\
Algorithmique\logiciels{}                       & L1      & TP     & 9      & 9    & 9    & 12    & 39\\
Bureautique\supports{}                          & L1, L3  & TP     & 16     & 12   & ---  & ---   & 28\\
Tutorat académique de stage           & L3      & ---    & 4      & ---  & ---  & ---   & 4\\\hline
\rowcolor[gray]{.9}\textbf{Total (éq. TD)}  &   &  & \textbf{65,5} & \textbf{67}   & \textbf{63} & \textbf{97,5}  & \textbf{293}\\\hline
\end{tabular}
\end{center}}

% (-2) vieux hack pr faire marcher colortbl et multirow !! http://les-mathematiques.u-strasbg.fr/phorum5/read.php?10,456219
\newcommand{\recapEnseignements}{%
\begin{center}\small %>{\centering\bf\arraybackslash\columncolor[gray]{.9}}
\begin{tabular}{|l|c|*{4}{>{\centering\arraybackslash}p{1.3cm}|}|>{\centering\bf\arraybackslash\columncolor[gray]{.9}}p{1.3cm}|}\hline
\multirow{2}{*}{\textbf{Intitulé}} & \multirow{2}{*}{\textbf{Niveau}} & 
\multicolumn{3}{c|}{\textbf{Moniteur}} & \textbf{\nicefrac{1}{2} ATER} & 
\\\cline{3-6}%\hhline{~~----~}
&  & \textbf{05--06} & \textbf{06--07}     & \textbf{07--08} & \textbf{08--09} & \multirow{-2}{*}{\textbf{Total}}\\\hline
Programmation orientée objet (Java)         & L3   & 11   & 11   & 23   & 23  & 68\\
Conception BD (Oracle)                      & L2, L3 & 12   & 8    & 8    & 26  & 54\\
Concepts objet avancés (Java)               & L3   & ---  & ---  & 23   & 23  & 46\\
Algorithmique (CompAlgo)                    & L1   & 9    & 9    & 9    & 12  & 39\\
Conception BD (Access)                      & L3   & 13,5 & 13,5 & ---  & --- & 27\\
Internet, Shell, Bureautique (OOffice)      & L1   & 16   & ---  & ---  & --- & 16\\
Bureau d'études BD (Oracle)                 & L3   & ---  & 13,5 & ---  & --- & 13,5\\
Tableur avancé (Excel)                      & L3   & ---  & 12   & ---  & --- & 12\\
Bases de données avancées (Oracle)          & M1   & ---  & ---  & ---  & 8   & 8\\
Intéropérabilité d'applications (XML)       & M1   & ---  & ---  & ---  & 5,5   & 5,5\\
Tutorat académique de stage (IUP ISI)       & L3   & 4    & ---  & ---  & --- & 4\\\hline
\rowcolor[gray]{.9}\textbf{Total (éq. TD)}  &   & \textbf{65,5} & \textbf{67}   & \textbf{63} & \textbf{97,5}  & \textbf{293}\\\hline
\end{tabular}
\end{center}}

\newcommand{\entryDescription}[1]{\hfill\mbox{#1\ \ }}

\newenvironment{Description}[1]%
	{\begin{list}{}{\renewcommand{\makelabel}{\entryDescription}%
		\settowidth{\labelwidth}{\ \ \ #1\ \ }%
		\setlength{\topsep}{\topsep+5mm}%
		\setlength{\leftmargin}{\labelwidth+\labelsep}}}%
	{\end{list}}
\newcommand{\exerg}[1]{\textbf{#1}}

\newcommand{\ueve}{Université d'Évry-Val d'Essonne\xspace}
\newcommand{\upn}{Université Paris-Nord\xspace}
\newcommand{\cnam}{Conservatoire National des Arts et Métiers\xspace}
\newcommand{\ups}{Université Paul-Sabatier\xspace}
\newcommand{\upmc}{Université Pierre-et-Marie-Curie\xspace}
\newcommand{\ue}{Université d'Évry\xspace}
\newcommand{\upd}{Université Paris-Dauphine\xspace}
\newcommand{\lycee}{Lycée Blaise-Pascal\xspace}
\newcommand{\cfd}{Clermont-Fd\xspace}
\newcommand{\isae}{Institut Supérieur de l'Aéronautique et de l'Espace\xspace}

\newcommand{\nom}{Fournier\xspace}
\newcommand{\nomusage}{Fournier-S'niehotta\xspace}
\newcommand{\prenom}{Raphaël\xspace}
\newcommand{\labo}{Laboratoire CEDRIC\xspace}
\newcommand{\ufr}{EPN5 Informatique\xspace}
\newcommand{\adrlabo}{Bureau 37-1-40\xspace}
\newcommand{\univ}{\cnam}
\newcommand{\ed}{EDITE\xspace}
\newcommand{\adruniv}{2, rue Conté 75003~Paris}
\newcommand{\labothese}{LIP6\xspace}
\newcommand{\univthese}{\upmc}
\newcommand{\mentionthese}{Très honorable\xspace}
\newcommand{\adrunivthese}{
4, place Jussieu\\
75252~Paris Cedex 5\\
\xspace}
\newcommand{\telwork}{+33 1 58 80 86 35\xspace}
\newcommand{\mailwork}{fournier@cnam.fr\xspace}
\newcommand{\webwork}{\url{http://raphael.fournier-sniehotta.fr}\xspace}

\newcommand{\premier}{$1^{\text{er}}$\xspace}
\newcommand{\sixieme}{$6^{\text{\`eme}}$\xspace}

